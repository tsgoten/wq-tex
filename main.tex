\documentclass{article}
\author{SBHS Science Olympiad, Tarang}
\usepackage[margin= 0in, landscape]{geometry}
\setlength{\columnsep}{.05in}
\usepackage{amsmath, multicol}
\usepackage{xcolor}
\usepackage{graphicx, wrapfig, float}

% Symbols
    \newcommand{\ddd}{$\bullet$}

% Colors
    \newcommand{\red}[1]{\textcolor{red}{#1}}
    \newcommand{\green}[1]{\textcolor{green}{#1}}
    \newcommand{\blue}[1]{\textcolor{blue}{#1}}
    \newcommand{\pink}[1]{\textcolor{magenta}{#1}}
    \newcommand{\orange}[1]{\textcolor{orange}{#1}}
    \newcommand{\yellow}[1]{\textcolor{yellow}{#1}}
% Headings
    % Note: The order of importance is the ROYGBV
    \newcommand{\mysection}[1]{\colorbox{yellow}{\textbf{\textit{\red{#1}}}}}
    \newcommand{\mysubsection}[1]{\underline{\textbf{{\textit{\orange{#1}}}}}}
    \newcommand{\mysubsub}[1]{{{\green{#1}}}}
    \newcommand{\mysubsubsub}[1]{{{\blue{#1}}}}
    \newcommand{\vocab}[1]{{\pink{#1}}}
% Pictures
\newcommand{\fig}[1]{
	\includegraphics[width=\columnwidth]{#1}
}
\newcommand{\figwidth}[2]{
	%file, width
	\includegraphics[width=#2cm]{#1}
}
\newcommand{\figwrap}[4]{
	%file, width, height, side
	\begin{wrapfigure}[#3]{#4}[0pt]{#2cm}
		\includegraphics[width=\linewidth]{#1}
	\end{wrapfigure}
}
    
\begin{document}
% Uncomment the line below to modify the font size
\tiny
\begin{multicols*}{4}
    \mysection{General Ecology} 
        \pink{Ecology}: how organisms interact with one another and with their environment. 
        \pink{Environment}: abiotic and biotic features
        \mysubsection{\textit{Levels of Organization:}} Population (same species) $>$  Community (diff. species, biotic) $>$ Ecosystem (community+abiotic) $>$ Biosphere (portion of Earth w/living species)
        \mysubsection{Productivity}
        \green{NPP} = rate at which energy is stored as biomass by plants or other primary producers and made available to consumers in ecosystem; \green{GPP} = rate at which solar energy is captured in sugar molecules.
        \ddd NPP = GPP - metabolism \ddd Net production efficiency = NPP/GPP
        \mysubsection{trophic level efficiency}: ratio of production of trophic level to lower level; green plants: 1-3\% of solar energy; herbivores: $ <1\% $; $ 10\% $ is a lie because it only applies to managed eco efficieny; usually, over $ 90\% $ of energy transfer between t-levels is lost as heat.
        \ddd \pink{ Microcosm: } small community 
        \ddd \pink{ Competition: } intraspecific: same species; inter: diff; 
        \ddd \pink{ Competitive exclusion: } 2 species competing for same limiting resources can’t coexist in same place;
        \ddd Ecological niche: sum of a species’ use of resources; 
        \ddd Resource partitioning: allows ecologically similar species to coexist in a community if they have significant differences in niches; 
        \ddd Fundamental vs Realized niche: potentially/actually occupied by a species; 
        \ddd Character displacement: tendency for characteristics to diverge in more in sympatric (no physical separation) pops than in allopatric pops; 
        \mysubsection{ Predation: (+/-) }
        \pink{Cryptic coloration: } camouflage, \pink{ Aposematic coloration: } warning colors, \pink{ Batesian mimicry: } harmless species mimics harmful, \pink{ Müllerian mimicry: } multiple unsavory species resemble each other, 
        \mysubsection{Herbivory: (+/-)} Organism eats plant or alga, defense: toxins, spines/thorns; 
        \mysubsection{Parasitism: (+/-):} 
        \ddd Endoparasites: Live in host, 
        \ddd Ectoparasites: Live outside host, .33<species on earth; 
        \mysubsection{Mutualism: (+/+)}
         Obligate mutualism: one species has lost ability to survive without its partner, Facultative mutualism: both species can survive alone if necessary; \pink{ Commensalism: (+/0 }); \pink{ Amensalism: (-/0) } 
    \mysection{Diversity}
		\ddd \vocab{ Alpha diversity } - diversity of each site \vocab{ Beta } - differences in species composition among sites \vocab{ Gamma } - diversity of entire landscape (regional species pool) EX: Yellowstone: Wolf eradication -> inc in elk herds -> dec in willows -> widening of river flow \ddd \vocab{ Species Richness: } count of species
		\ddd \vocab{ Simpson Index: } $ D = [\sum n(n-1)]/ [N(N-1)] $ probability that two individuals randomly selected from a sample will belong to same species, bigger value of D, LOWER diversity; ranges from 0-1 
		\ddd \vocab{ Simpson Index of Diversity (1 - D): } greater value, greater sample diversity, probability that two individuals randomly selected from a sample will belong to different species; 
		\ddd \vocab{ Reciprocal Index 1 / D: } 1 = community containing only one species, higher value, greater diversity, maximum value = number of species in sample 
		\ddd \vocab{ Shannon-Weiner Index: } $ p_i $ is often proportion of individuals belonging to ith species in dataset of interest, quantifies uncertainty in predicting species identity of an individual that is taken at random from dataset $ H = - \sum [P_i \times \ln(P_i)] $ (for each species) $ P_i $ = sample/sum | Evenness = H/Hmax | Hmax = ln(S) where S is species richness(count) 
		\ddd \vocab{ Sorensen’s Coefficient (CC) } =  2C/(S1+S2) . Where C is the number of species the two communities have in common, S1 is the total number of species found in community 1, and S2 is the total number of species found in community 2. 
		\ddd \vocab{ Effective population } $ N_e = 4*N_m*N_f / N_m + N_f $ 
		\ddd \vocab{ Cohort Life Tables } - $ l_x $ is survivorship (prop. That survived to next stage), $ d_x $ is mortality (prop. that die during stage), $ q_x $ is mortality rate, $ l_xm_x $ is avg number of offspring per female or contribution of each age class to overall, pop. growth rate, $ l_xm_x,x $ = weighs $ l_xm_x $ by life stage, $ R_0 $ = sum of $ l_xm_x $ = reproductive rate, or the average net number of offspring produced by an individual in its lifetime.
    \mysection{Stream Ecology}
        \mysubsection{Watershed} (drainage basin, catchment area): a land area that channels rainfall and snowmelt to creeks, streams, and rivers, and eventually to outflow points such as reservoirs, bays, and the ocean
        \mysubsection{Riparian Zone} narrow area alongside a stream that has its own special vegetation; contributes nutrients, shade, organic materials for small organisms, soil stability, habitat
        \mysubsection{Keystone Species}:  species whose functions are so intertwined with the lives of other animals that their removal can cause imbalance or collapse
        \mysubsection{Stream Anatomy} 
        	\mysubsub{Edgewater: } habitats may have emergent plants, sheltered overhangs with suspended root mats and leaf packs in quiet back eddies. The composition of macro-invertebrates will tend to differ from that in riffles. Animals survive best in places that provide protection, camouflage and food sources.         	
        	\mysubsub{Riffles}: Shallow with fast, turbulent water running over rocks. Only animals that cling very well, such as net-winged midges, caddisflies, stoneflies, some mayflies, dace, and sculpins can spend much time here, and plant life is restricted to diatoms and small algae. Riffles are a good place for mayflies, stoneflies, and caddisflies to live because they offer plenty of cobbly gravel to hide in.
        	\mysubsub{Runs}: Close to any pool or riffle is a run, which merely describes a main body of water that runs smoothly downstream. Fishes, like minnows, too small to compete for pools often end up in runs.
        	\mysubsub{Pools}: When a stream meets up with a huge fallen log, or a set of boulders, the water pours over the top. The vertical force of the water falling down on the other side will carve out a pool in the stream. Preferred by trouts, mollusks (like clams and snails) and worms. Benefits to slow-moving water is that organic debris settles out into it. Also you don't have to relocate to another area if the stream level starts to lower.
        \mysubsection{Floods}
            \ddd Riparian zones depend on floods
            \mysubsub{Adaptions to flooding}: fishes wait for annual spring flood to start breeding, insect larvae lay eggs, hatch, or metamorphose, new food sources, increased fertility
        \mysubsection{Dams}
            changes ecology \textbf{forever}, habitats removed, fish die from turbines
            \mysubsub{Hydroelectric Stations Impact}
                Silt Loads; Water Temperatures; River Flow; Dissolved Oxygen. 
            \mysubsub{Channelizing Streams}
                \ddd Done to protect property and roads from flooding
                \ddd Stream becomes poor in nutrients and habitat; without periodic flooding, riparian zone dies, native fish die
                \ddd More channelization also decreases control over river, and erosion threatens buildings
                \ddd When a stream is allowed to meander, it pushes against banks and swirls, reducing energy of water, but when streams are channelized straight down a mountain, it has more energy
        \mysubsection{Development}
            \ddd \vocab{Urban} runoff - more oil, fertilizer, pesticides, herbicides end up in streams
            \ddd Lack of trees - takes away shade from stream, warming it up; bugs that fuel food chain in stream will no longer fall into stream
            \ddd lack of roots of vegetation will cause soil to erode away into the river (as it is no longer grounded in place with vegetation) 
        \mysubsection{Logging}
            \ddd Silt clouds up rivers
            \ddd Silt that settles in the bottom of the river prevents eggs of some species (like Salmon and Trout) that reside in gravel in bottom of river from receiving dissolved oxygen from flowing water above
            \ddd The silt in between the gravel also destroys the habitat of many aquatic insects, and takes away the food source of fish as well
        \mysubsection{Mining}
            \ddd introduces heavy metal and radioactive waste into river
            \ddd makes river acidic
            \ddd Requires much water and reduces level of aquifer, drying streams
    \mysection{Lake Terminology}
        \mysubsection{Olgotrophic} clear water, low productivity, very good fishery of large game fish
            \ddd Deep, nutrient poor lakes in which the phytoplankton is not very productive. Deep zone has high [O2] since there is very little detritus
            \ddd Can develop into eutrophic over time
            \ddd Runoff brings in mineral nutrients and sediments
            \ddd Human activities increase nutrient content of runoff due to fertilizers
            \ddd Municipal wastes dumped into lakes enriches N and P, so more phytoplankton
            \ddd Algal blooms and increased plant growth creates more detritus and can lead to oxygen depletion.
        \mysubsection{Mesotrophic} increased production, accumulated organic mater, occasional algal bloom, good fishery
        \mysubsection{Eutrophic} very productive, can experience oxygen depletion, rough fish common
        \mysubsection{Eutrophication} 
        	The process by which a body of water develops a high concentration of nutrients, like nitrates and phosphates. Phosphorus and Nitrogen are both released from sources related to land use. Nutrients cause an increase in blue-green bacteria and algae, which can cover the surface of the lake in mats; this blocks sunlight to plants; as algae and bacteria die, they decompose and BOD increases, decreasing DO. If lowered enough, fish will die. 
    \mysection{Aquatic Ecosystems}
        \mysubsection{Lake Zones}
            \mysubsection{Lentic Ecosystems} (STILL Water) 
            \mysubsub{Ponds} Bottom of the pond still receives light, unlike lakes.
            \mysubsub{Horizontal Lake Zones}
                \mysubsubsub{Littoral Zone}: Near the shoreline; Sunlight penetrates all the way to sediments; Allows for aquatic plants (\vocab{macrophytes}) to grow.
                \mysubsubsub{Limentic Zone} open water, away from shore.
            \mysubsub{Vertical Lake Zones}
                \mysubsubsub{Photic}
                    Depth in which photosynthesis can occur.
                \mysubsubsub{Aphotic}
                Photosynthesis cannot occur; Most organisms are invertebrates. Productivity depends on the organic content of the sediment.
          \mysubsection{Lake Turnover}
               heated by the sun The deepest layer, the hypolimnion, is the coldest. The sun's radiation does not reach this cold, dark layer. 
        	  \ddd During the fall, the warm surface water begins to cool. As water cools, it becomes more dense, causing it to sink. This dense water forces the water of the hypolimnion to rise, \textbf{"turning over"} the layers. The opposite happens during the spring.
        	  \mysubsection{Lotic Ecosystems (Flowing Water)}
        	  \vocab{Stream Order} or waterbody order is a positive whole number used in geomorphology and hydrology to indicate the level of branching in a river system. 
        \mysubsection{Mixing and stratification}
            \ddd Exhibit significant vertical stratification with light penetration and temperature 
            \ddd Light penetration stratification - Ponds or lakes are divided into two layers due to a decrease in light intensity with increasing depth - as light is absorbed by the water and suspended microorganisms.
            \ddd \blue{Photic zone} upper layer where light is sufficient for photosynthesis
            \ddd \blue{Aphotic zone} bottom layer with little light, no photosynthesis
            \ddd For deeper ponds and lakes, temperature stratification occurs; sunlight warms the upper layer as far as it can penetrate
            \ddd \blue{Thermocline} - narrow vertical zone between warm and cold layers where a rapid temp change occurs 
	\mysection{Cycles}
		\mysubsection{Oxygen}
		Three main reservoirs: \pink{ atmosphere (air) }, total content of \pink{ biological matter } within biosphere, \& \pink{ Earth's crust }. Failures in O-cycle w/in hydrosphere (combined mass of water found on, under, \& over surface) can result in development of \pink{ hypoxic zones }. Main driving factor of O-cycle is photosynthesis, which is responsible for modern Earth's atmosphere \& life. 
		\mysubsection{Nitrogen}
			\mysubsub{Fixation: } 
				done by free-living (ex. Azotobacter) or symbiotic bacteria known as diazotrophs, which have nitrogenase enzyme that combines gaseous nitrogen with hydrogen to produce ammonia (which is converted into other organic compounds by bacteria). Most biological nitrogen fixation occurs by activity of Mo-nitrogenase (a complex two component enzyme that has multiple metal-containing prosthetic groups), found in variety of bacteria \& some Archaea. Symbiotic nitrogen-fixing bacteria (ex. Rhizobium) usually live in root nodules of legumes (peas, alfalfa, \& locust trees) \& a few non-legumes, forming a mutualistic relationship with plant, producing ammonia in exchange for carbohydrates. Because of this, legumes often increase nitrogen content of nitrogen-poor soils.
			\mysubsub{Assimilation: }
				Plants take nitrogen from soil by absorption through their roots via root hairs as amino acids, nitrate ions, nitrite ions, or ammonium ions. If nitrate is absorbed, it is first reduced to nitrite ions \& then ammonium ions for incorporation into amino acids, nucleic acids, \& chlorophyll. Plants that have a symbiotic relationship with rhizobia assimilate some nitrogen in form of (NH4+) directly from nodules.
			\mysubsub{Ammonification: }
				When a plant/animal dies or an animal expels waste, nitrogen is initially organic. Bacteria or fungi convert organic N w/in remains back into ammonium (NH4+), a process called ammonification or mineralization
			\mysubsub{Denitrification: }
				eduction of nitrates back into nitrogen gas (N2), completing N cycle, performed by bacterial species such as Pseudomonas \& Clostridium in anaerobic conditions. Nitrate used as an electron acceptor in place of oxygen during respiration. These facultatively anaerobic bacteria can also live in aerobic conditions. Denitrification happens in anaerobic conditions e.g. waterlogged soils. Denitrifying bacteria use nitrates in soil to carry out respiration \& consequently produce N2, which is inert \& unavailable to plants 
			\mysubsub{Nitrification: } 
				Conversion of ammonium to nitrate performed primarily by soil-living \& other nitrifying bacteria. In primary stage, oxidation of ammonium (NH4+) is performed by bacteria such as Nitrosomonas species, which converts ammonia to nitrites (NO2). Other bacterial species, such as Nitrobacter, are responsible for oxidation of nitrites into nitrates (NO3). Ammonia conversion to nitrates or nitrites is important because ammonia gas is toxic to plants.
		\mysubsection{Phosphorus}
 			Essential nutrient for plants \& animals essential in form of ions PO3-43- \& HPO2-4, part of DNA, ATP, ADP, fats of cell membranes, building block of certain parts of animal \& human body; doesn’t enter atmosphere, remains mostly on land \& in rock \& soil minerals; liquid at normal temp \& pressure; cycling through water, soil, \& sediments; slow matter cycle; Moves slowly from deposits on land \& in sediments, to living organisms, \& turn much more slowly back into soil \& water sediment. Passes through plants \& animals much faster than through rocks \& sediments. Animals get it by eating plants or plant-eating animals; When plants \& animals die, phosphates return soils or oceans, ending up in sediments or rock formations again, remaining there for millions of years. Eventually, it is released through weathering \& it begins again; Most commonly found in rock formations \& ocean sediments as phosphate salts. Phosphate salts released through rocks usually dissolve in soil water \& will be absorbed by plants; Quantities of phosphorus in soil are small, often making them limiting factors for plant growth. Not very water-soluble, making them limiting factors for plant growth in marine ecosystems; Constant additions of phosphates by humans \& exceeding natural concentrations disrupts P-cycle strongly; Phosphate can build up in rivers \& lakes, causing excessive algae growth; Increasing phosphorus concentrations in surface waters raise growth of phosphate-dependent organisms, like algae \& duckweed. These organisms use great amounts of oxygen \& prevent sunlight from entering water, known as eutrophication.
   	\mysection{Wastewater treatment}
    	\mysubsection{Potable Water Treatment}
    		\textbf{1} \mysubsub{Coagulation and Flocculation}: First step. Chemicals with a positive charge are added to the water to neutralize the negative charge of dirt and other dissolved particles. Then the particles bind with the chemicals and form larger particles, called \pink{floc}.
    		\textbf{2} \mysubsub{Sedimentation}: floc settles to the bottom of the water supply, due to its weight.
    		\textbf{3} \mysubsub{Filtration}: Once the floc has settled to the bottom of the water supply, the clear water on top will pass through filters of varying compositions (sand, gravel, and charcoal), in order to remove dissolved particles (dust, parasites, bacteria, viruses, chemicals).
    		\textbf{4} \mysubsub{Disinfection}: a disinfectant (chlorine, chloramine) may be added in order to kill any remaining parasites, bacteria, and viruses, and to protect the water from germs when it is piped to homes and businesses.
        \mysubsection{Groundwater treatment techniques}
             \ddd \vocab{Air sparging:} inject oxygen into groundwater; When used in combination with soil vapor extraction (SVE), air bubbles carry vapor phase contaminants to a SVE system which removes them. 
             \vocab{\ddd Bioreactors: }vessel in which a chemical process is carried out which involves organisms or biochemically active substances derived from such organisms Chemical Oxidation 
             \ddd \vocab{Constructed wetland: } artificial wetland to treat municipal or industrial wastewater, greywater or stormwater runoff. It may also be designed for land reclamation after mining, or as a mitigation step for natural areas lost to land development. use natural functions vegetation, soil, and organisms to treat wastewater. Acts as a biofilter and removes pollutants.
             \ddd \vocab{Dual phase extraction: } uses a high-vacuum system to remove both contaminated groundwater and soil vapor. 
             \ddd \vocab{Pump and treat: } treatment of pumped groundwater before it is released. 
             \ddd \vocab{Phytoremediation: }direct use of living green plants for in situ, or in place, removal, degradation, or containment of contaminants in soils, sludges, sediments, surface water and groundwater. 
        \mysubsection{Septic-Tank Disposal Systems: }
            A sewer line from the house leads to an underground septic tank designed to separate solids from liquid, digest and store organic matter, and allow the treated sewage to seep into the surrounding soil. As the wastewater moves through the soil, it is further treated by the natural processes of oxidation and filtering.
            \ddd This method can \textbf{fail} if the tank isn't pumped out when it's full of solids or if there is poor drainage in the surrounding soil.
        \mysubsection{Primary Treatment} 
            removes 30-40\% of BOD by volume, mainly in the form of suspended solids and organic matter. 
            \ddd Incoming raw sewage passes in and is first passed through a series of screens to remove large floating organic material
            \ddd Sewage next enters the grit chamber, where sand, small stones, and grit are removed
            \ddd Then primary sedimentation tank, where particulate matter settles out to form a sludge. Chemicals can be used to help the settling process
            \ddd Sludge is removed and transported for further processing
    	 \mysubsection{Secondary Treatment}
    	    \ddd \vocab{Activated sludge} (most common treatment)
    	    \ddd \vocab{Aeration tank} wastewater is pumped with air and some sludge from final sedimentation tank; sludge contains aerobic bacteria that consume organic material; the wastewater enters the final sedimentation tank and sludge settles out; some of activated sludge is recycled with new air and wastewater; the rest of the sludge goes to digester, where \vocab{anaerobic bacteria} further break down sludge;
    	    \ddd $CH_4$ is a product of this anaerobic breakdown, and is captured for fuel. 
    	    \ddd Wastewater is finally disinfected to remove pathogens, usually by chlorination
    	  \mysubsection{Advanced Wastewater Treatment}
    	    There may still be nutrients, organic chemicals, and heavy metals that can be removed with sand filters, carbon filters, and special chemicals        
        \mysubsection{Septic Tank: } 
        	The sewer line from the house leads to an underground septic tank, which is designed to separate solids from liquid, treat, and store organic matter through a period of detention, and allow the clarified liquid to discharge into the drain (absorption) field from piping through which the treated sewage seeps into the surrounding soil. As the wastewater moves through the soil, it is further treated by natural processes of oxidation and filtering. These may fail bc of failure to pump out the septic tank when it is full of solids, and poor soil drainage, which allows the effluent to rise to the surface in wet weather
    \mysection{Analysis}
        Salinity, pH, alkalinity, phosphates, nitrates, turbidity, dissolved oxygen (DO), temperature, fecal coliform, total solids, and biological oxygen demand (BOD). 
        \mysubsection{Salinity}
            Water can be classified by its salinity as such: \textbf{fresh water has a ppt of $< 0.5$} which means that there are 0.5 molecules of dissolved salt for every 1000 molecules of solution, or 1 molecule of salt per every 2000 molecules of solution. Brackish water has a ppt between 0.5 and 30, saline water has a ppt between 30 to 50, and brine has a ppt of >50. The only water safe for human consumption is fresh water, and drinking water often achieves salinity levels as low as 0.1 ppt. In ocean water, total salt content makes up 3.5\% (35 ppt) of ocean water, the other 96.5\% being water. Of the dissolved salts in ocean water, 85\% of the salts is sodium chloride (3\% of all the water, 30 ppt). The other 15\% of the total salts are other salt ions such as Magnesium, Strontium etc. (.5 \% of total ocean water, 5 ppt). 
            \mysubsubsub{Consequences}
                Salinity in rivers and lakes of the US has recently been increasing due to road salt and other salt de-icers in runoff. \ddd It is very expensive to remove salt from water, and thus it is expensive to create drinking wate
        \mysubsection{pH}
            $pH = - \log [H^+]$
            \ddd  The normal pH of rivers in the United States is 6.5 to 8.5, and values between 6.0 and 9.0 can support life for fish and invertebrates.
            \mysubsub{Influenced by} 
            \ddd \textbf{Human processes} like automobile/fossil fuel power plant emissions release nitrogen oxides and sulfur dioxide.
            \ddd  Coal mine drainage can lead to iron sulfide mixing with water.
            \ddd \textbf{Natural}. Limestone is a base when dissolved in water, so it can neutralize the effects of acids and increase the pH of the water. olcanoes, geysers, and hot springs will make water more acidic, as well as the presence of sulfur in nearby minerals.
        \mysubsection{Alkalinity}
            The ability of a solution to neutralize an acid without changing the overall pH. From presence of buffers. 
            \mysubsub{Effects} If \textbf{alkalinity is too low}, the ecosystem has low stability as it is susceptible to sudden pH changes from devices such as acid rain or other pollution, which can be harmful to the flora and fauna of the ecosystem. If \textbf{alkalinity is too high}, the buffer acids and bases in the buffer solution can render the ecosystem uninhabitable.
        \mysubsection{Phosphates}
            Phosphates can become a limiting nutrient in many systems, usually freshwater systems. Generally, the only \textbf{negative effect} of an overabundance of phosphates is \pink{eutrophication}. Phosphates can contribute to total dissolved solids. Extremely high levels of phosphates in \textbf{drinking water} can cause digestive issues. 
        \mysubsection{Nitrates}
           Nitrates can be a limiting nutrient, usually in marine systems rather than freshwater systems. However, high levels of nitrates in aquatic ecosystems can be detrimental to ecosystem health, inhibit the growth of some organisms, cause stress, and contribute to eutrophication. Nitrates do contribute to total dissolved solids and can be used as a water quality indicator. 
        \mysubsection{Turbidity}
            \ddd Increased salinity leads to low turbidity because salt ions bunch suspended solids in the water together, making it more easy for suspended solids to settle at the bottom. So oceans and estuaries tend to be more clear than freshwater. 
            \ddd Turbidity is a measurement of how much light the water scatters, so it gets affected by not only suspended solids, but also colored dissolved materials, or dyes in the water.
            \ddd \vocab{“Turbidity maximum zone”: }caused by change in flow when freshwater stream enters saltwater estuary. \pink{Clear} = under 20 mg/L (of total suspended solids) \pink{Cloudy} = over 40 mg/L $ \uparrow $ temperature, $ \downarrow $ DO; $ \uparrow $ temperature, $ \uparrow $ stratification $ \rightarrow $ may create hypoxic zones in lower levels because this is where most decomposition occurs; harmful to living organisms $ \uparrow $; turbidity, $ \downarrow $ photosynthesis and plant growth,  $ \downarrow $ DO (b/c less plants and decomposition causes loss of DO), $ \downarrow $ primary food source in food chain, $ \downarrow $ pop. Of org,
        \mysubsection{Dissolved Oxygen}
            \ddd Surface waters contain between \pink{5 and 15 ppm} of dissolved oxygen. If a stream or river has below 5 ppm of dissolved oxygen, then that can put aquatic life under stress, and below 1-2 ppm for a few hours can kill large fish living in the river.
            \ddd Supersaturation of oxygen (levels of DO over 100\%) can occur naturally through photosynthetically active species. Supersaturation can also occur through rapid changes in the environment that occur too quickly for the system to reach equilibrium, giving rise to DO levels over 100\% temporarily.
            \ddd < 6 mg/L harmful to pond life
            \ddd As temperature increases, DO decreases
            \ddd Oxygen level changes result from streamflow, temperature, and run-off
            \ddd At sea level, \textbf{\blue{typical DO concentrations}} in 100-percent saturated fresh water will range from $7.56 mg/L$ (or 7.56 parts oxygen in 1,000,000 parts water) at $30^\circ C$ to $14.62 mg/L$ at zero $0^\circ C$.
            \ddd Measured by the \vocab{Winkler Test}
        \mysubsection{Biochemical Oxygen Demand}
            \ddd (BOD) measures how fast organisms use up the oxygen in the water. Aerobic microbes use oxygen to oxidize the organic matter in the water, using the energy that is released in the process for growth and reproduction and creating a demand for DO.
            \ddd Most pristine rivers should have a 5-day BOD below 1 mg/L. Moderately polluted rivers may have a BOD between 2 and 8 mg/L. Municipal sewage that is treated with a three-stage process would have a BOD of about 20 mg/L. Untreated sewage has varying BOD but averages about 200 mg/L    
        \mysubsection{Temperature}
             \ddd Affects the amount of gases such as oxygen that can be dissolved in the water – \textbf{cold water holds more oxygen than warm water}$ \rightarrow $Increases the metabolic rates of aquatic organisms. 
             \ddd Affects the rate of photosynthesis by aquatic plants and algae. 
             \ddd Increases the sensitivity of organisms to disease, parasites and pollution. 
             \ddd Small chronic temperature changes can adversely affect the reproductive systems of aquatic organisms. 
             \ddd Raising water temperature increases decomposition rate of organic matter, depleting DO. \ddd Types of temperature changes include natural seasonal changes, man’s activities, industrial thermal pollution as discharge of cooling water, stormwater runoff from heated surfaces as streets, roofs, parking lots, soil erosion increasing water turbidity which warms the water, removal of shade trees from along the shores.
        \mysubsection{Total Solids} 
        	Dissolved solids in the body of water and is subdivided into two categories: total suspended solids (TSS) and total dissolved solids (TDS).
\end{multicols*}
\end{document}