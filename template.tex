\documentclass{article}
\author{SBHS Science Olympiad, Tarang}
\usepackage[margin=.05in, landscape]{geometry}
\setlength{\columnsep}{.1in}
\usepackage{amsmath, multicol}
\usepackage{xcolor}
\usepackage{graphicx, wrapfig, float}


% Symbols
    \newcommand{\ddd}{$\bullet$}

% Colors
    \newcommand{\red}[1]{\textcolor{red}{#1}}
    \newcommand{\green}[1]{\textcolor{green}{#1}}
    \newcommand{\blue}[1]{\textcolor{blue}{#1}}
    \newcommand{\pink}[1]{\textcolor{magenta}{#1}}
    \newcommand{\orange}[1]{\textcolor{orange}{#1}}
    \newcommand{\yellow}[1]{\textcolor{yellow}{#1}}
% Headings
    % Note: The order of importance is the ROYGBV
    \newcommand{\mysection}[1]{\textbf{\textit{\red{#1}}}}
    \newcommand{\mysubsection}[1]{{\textit{\orange{#1}}}}
    \newcommand{\mysubsub}[1]{{{\green{#1}}}}
    \newcommand{\mysubsubsub}[1]{{{\blue{#1}}}}
    \newcommand{\vocab}[1]{{\pink{#1}}}
% Pictures
\newcommand{\fig}[1]{
	\includegraphics[width=\columnwidth]{#1}
}
\newcommand{\figwidth}[2]{
	%file, width
	\includegraphics[width=#2cm]{#1}
}
\newcommand{\figwrap}[4]{
	%file, width, height, side
	\begin{wrapfigure}[#3]{#4}[0pt]{#2cm}
		\includegraphics[width=\linewidth]{#1}
	\end{wrapfigure}
}
    
\begin{document}
% Uncomment the line below to modify the font size
\tiny
\begin{multicols*}{4}
    \noindent
    \mysection{General Ecology} \\
        Ecology: how organisms interact with one another and with their environment \\
        Environment: abiotic and biotic features\\
        \textit{Levels of Organization:} Population (same species) $>$  Community (diff. species, biotic) $>$ Ecosystem (community+abiotic) $>$ Biosphere (portion of Earth w/living species)
        \mysubsection{Productivity}
        \green{NPP} = rate at which energy is stored as biomass by plants or other primary producers and made available to consumers in ecosystem; \green{GPP} = rate at which solar energy is captured in sugar molecules.
        \ddd NPP = GPP - metabolism \ddd Net production efficiency = NPP/GPP
    \\
    \mysection{Stream Ecology}
        \mysubsection{Watershed} (drainage basin, catchment area): a land area that channels rainfall and snowmelt to creeks, streams, and rivers, and eventually to outflow points such as reservoirs, bays, and the ocean
        \mysubsection{Riparian Zone} narrow area alongside a stream that has its own special vegetation; contributes nutrients, shade, organic materials for small organisms, soil stability, habitat
        \mysubsection{Keystone Species}:  species whose functions are so intertwined with the lives of other animals that their removal can cause imbalance or collapse
        \mysubsection{Floods}
            \ddd Riparian zones depend on floods
            \mysubsub{Adaptions to flooding}: fishes wait for annual spring flood to start breeding, insect larvae lay eggs, hatch, or metamorphose, new food sources, increased fertility
        \mysubsection{Dams}
            changes ecology \textbf{forever}, habitats removed, fish die from turbines
            \mysubsub{Hydroelectric Stations Impact}
                Silt Loads; Water Temperatures; River Flow; Dissolved Oxygen. 
            \mysubsub{Channelizing Streams}
                \ddd Done to protect property and roads from flooding
                \ddd Stream becomes poor in nutrients and habitat; without periodic flooding, riparian zone dies, native fish die
                \ddd More channelization also decreases control over river, and erosion threatens buildings
                \ddd When a stream is allowed to meander, it pushes against banks and swirls, reducing energy of water, but when streams are channelized straight down a mountain, it has more energy
        \mysubsection{Development}
            \ddd \vocab{Urban} runoff - more oil, fertilizer, pesticides, herbicides end up in streams
            \ddd Lack of trees - takes away shade from stream, warming it up; bugs that fuel food chain in stream will no longer fall into stream
            \ddd lack of roots of vegetation will cause soil to erode away into the river (as it is no longer grounded in place with vegetation) 
        \mysubsection{Logging}
            \ddd Silt clouds up rivers
            \ddd Silt that settles in the bottom of the river prevents eggs of some species (like Salmon and Trout) that reside in gravel in bottom of river from receiving dissolved oxygen from flowing water above
            \ddd The silt in between the gravel also destroys the habitat of many aquatic insects, and takes away the food source of fish as well
        \mysubsection{Mining}
            \ddd introduces heavy metal and radioactive waste into river
            \ddd makes river acidic
            \ddd Requires much water and reduces level of aquifer, drying streams
    \\
    \mysection{Lake Terminology}
        \mysubsection{Olgotrophic} clear water, low productivity, very good fishery of large game fish
            \ddd Deep, nutrient poor lakes in which the phytoplankton is not very productive. Deep zone has high [O2] since there is very little detritus
            \ddd Can develop into eutrophic over time
            \ddd Runoff brings in mineral nutrients and sediments
            \ddd Human activities increase nutrient content of runoff due to fertilizers
            \ddd Municipal wastes dumped into lakes enriches N and P, so more phytoplankton
            \ddd Algal blooms and increased plant growth creates more detritus and can lead to oxygen depletion.
        \mysubsection{Mesotrophic} icreased production, accumulated organic mater, ocassional algal bloom, good fishery
        \mysubsection{Eutrophic} very productive, can experience oxygen depletion, rough fish common
    \\
    \mysection{Aquatic Ecosystems} \\
        \mysubsection{Lake Zones}
            \mysubsection{Lentic Ecosystems} (STILL Water) \\
            \mysubsub{Ponds} Bottom of the pond still receives light, unlike lakes.
            \\
            \mysubsub{Horizontal Lake Zones}
                \mysubsubsub{Littoral Zone}: Near the shoreline; Sunlight penetrates all the way to sediments; Allows for aquatic plants (\vocab{macrophytes}) to grow.
                \mysubsubsub{Limentic Zone} open water, away from shore.
            \\
            \mysubsub{Vertical Lake Zones}
                \mysubsubsub{Photic}
                    Depth in which photosynthesis can occur.
                \mysubsubsub{Aphotic}
                Photosynthesis cannot occur; Most organisms are invertebrates. Productivity depends on the organic content of the sediment.
          \mysubsub{Lake Turnover}
               heated by the sun The deepest layer, the hypolimnion, is the coldest. The sun's radiation does not reach this cold, dark layer. 
        	  \ddd During the fall, the warm surface water begins to cool. As water cools, it becomes more dense, causing it to sink. This dense water forces the water of the hypolimnion to rise, "turning over" the layers.
        	  \figwidth{turnover}{5}
        	  \mysubsection{Lotic Ecosystems (Flowing Water)}
        	  \vocab{Stream Order} or waterbody order is a positive whole number used in geomorphology and hydrology to indicate the level of branching in a river system. 
             %\fig{streamorder}
        	 % \fig{OEPA-RCCpsd}
        \mysubsection{Mixing and stratification}
            \ddd Exhibit significant vertical stratification with light penetration and temperature 
            \ddd Light penetration stratification - Ponds or lakes are divided into two layers due to a decrease in light intensity with increasing depth - as light is absorbed by the water and suspended microorganisms.
            \ddd \blue{Photic zone} upper layer where light is sufficient for photosynthesis
            \ddd \blue{Aphotic zone} bottom layer with little light, no photosynthesis
            \ddd For deeper ponds and lakes, temperature stratification occurs; sunlight warms the upper layer as far as it can penetrate
            \ddd \blue{Thermocline} - narrow vertical zone between warm and cold layers where a rapid temp change occurs
    \\
	  \mysection{Water Cycle}
	    \ddd The oceans are the largest reservoir with about 97\% of all the water is too saline for most human use.
	    \ddd Ice caps and glaciers are largest reservoirs of freshwater
	    \ddd The water on land can either return to the ocean by surface runoff, rivers, glaciers, and subsurface groundwater flow, or return to the atmosphere by evaporation or transpiration (loss of water by plants to the atmosphere).
	    \ddd 35,000 mg of dissolved ions per liter of seawater.
	    \mysubsection{Precipitation}
	        \ddd Precipitation levels are unevenly distributed around the globe
	        \ddd Water quality could suffer in areas experiencing increases in rainfall.
	        \ddd increases in heavy precipitation events could cause problems for the water infrastructure, as sewer systems and water treatment plants are overwhelmed by the increased volumes of water. 
	        \ddd Sea level rise face risks to freshwater resources.
	 \\
    \mysection{Wastewater treatment}
        \mysubsection{Groundwater treatment techniques}
             1. Air sparging 2. Bioreactors 3. Chemical oxidation 4. Constructed wetlands 5. Dual phase extraction 6. Pump and treat 7. Phytoremediation
        \mysubsection{Septic-Tank Disposal Systems: }
            This is the conventional method for treatment. A sewer line from the house leads to an underground septic tank in the yard. This tank is designed to separate solids from liquid, digest and store organic matter, and allow the treated sewage to seep into the surrounding soil. As the wastewater moves through the soil, it is further treated by the natural processes of oxidation and filtering.
            \ddd This method can fail if the tank isn't pumped out when it's full of solids or if there is poor drainage in the surrounding soil.
        \mysubsection{Primary Treatment} 
            removes 30-40\% of BOD by volume, mainly in the form of suspended solids and organic matter. 
            \ddd Incoming raw sewage passes in and is first passed through a series of screens to remove large floating organic material
            \ddd Sewage next enters the grit chamber, where sand, small stones, and grit are removed
            \ddd Then primary sedimentation tank, where particulate matter settles out to form a sludge. Chemicals can be used to help the settling process
            \ddd Sludge is removed and transported for further processing
    	 \mysubsection{Secondary Treatment}
    	    \ddd \vocab{Activated sludge} (most common treatment)
    	    \ddd \vocab{Aeration tank} wastewater is pumped with air and some sludge from final sedimentation tank; sludge contains aerobic bacteria that consume organic material; the wastewater enters the final sedimentation tank and sludge settles out; some of activated sludge is recycled with new air and wastewater; the rest of the sludge goes to digester, where \vocab{anaerobic bacteria} further break down sludge;
    	    \ddd $CH_4$ is a product of this anaerobic breakdown, and is captured for fuel. 
    	    \ddd Wastewater is finally disinfected to remove pathogens, usually by chlorination
    	  \mysubsection{Advanced Wastewater Treatment}
    	    There may still be nutrients, organic chemicals, and heavy metals that can be removed with sand filters, carbon filters, and special chemicals
    \\
    \mysection{Nutrient Cycle}
    \\
        \mysubsection{Nitrogen Cycle}
            \ddd nitrogen in atmosphere turned into nitrates that fall to earth from lightning; this fertilizes plants and stimulates growth
            \ddd Ammonia and nitrates absorbed through roots of plants, which is ingested by animals
            \ddd Decomposers break down proteins from amino acids (containing nitrogen) from plants and animals 
            \ddd \vocab{Ammonification}: nitrogen from amino acids turned into ammonia from decomposers
            \ddd Ammonia is \vocab{nitrified} (turned into a nitrate) from the bacteria
            \ddd Through \vocab{denitrification}, bacteria converts ammonia and nitrate into nitrous oxide and nitrogen which is released into the atmosphere
    \\
        \mysubsection{Phosphorus Cycle}
            \ddd Differs from other cycles because \textbf{does not involve gas phase}
            \ddd When it rains, phosphates from sedimentary rocks are dispersed to water bodies and soil
            \ddd Phosphate ions in soil transferred to plants, herbivores and carnivores
            \ddd form Feces/urine of animals
            \ddd Not highly soluble; enters water through sedimentary runoff
            \ddd Also enters waterways through industrial waste
            \ddd Aquatic plants use water-born phosphates, which then travels up the food chain
            \ddd Excessive amounts may cause \vocab{eutrophication}(excessive richness of nutrients in a body of water) because phosphate stimulates plant growth. The plants take up too much oxygen and block sunlight, suffocating marine life. 
        \\
        \mysubsection{Oxygen}
            \ddd taken by animals for cellular respiration
            \ddd released by plants and protists through photosynthesis.
            \ddd Phytoplankton in oceans release most of the oxygen in the world.
        \mysubsection{Carbon}
            \ddd \vocab{Combustion}: Burning of wood and fossil fuels by factory and auto emissions transfers carbon to the atmosphere as carbon dioxide.
            \ddd \vocab{Photosynthesis}: Carbon dioxide is taken up by plants during photosynthesis and is converted into energy rich organic molecules, such as glucose, which contains carbon.
            \ddd \vocab{Metabolism}: Autotrophs convert carbon into organic molecules like fats, carbohydrates and proteins, which animals can eat.
            \ddd \vocab{Cellular respiration}: Animals eat plants for food, taking up the organic carbon (carbohydrates). Plants and animals break down these organic molecules during the process of cellular respiration and release energy, water and carbon dioxide. Carbon dioxide is returned to the atmosphere during gaseous exchange. 
            \ddd \vocab{Precipitate}: Carbon dioxide in the atmosphere can also precipitate as carbonate in ocean sediments.
            \ddd \vocab{Decay}: Carbon dioxide gas is also released into the atmosphere during the decay of all organisms.
        \mysubsection{Dissolved Oxygen}
            \ddd As temperature increases, DO decreases
            \ddd Different organisms adapt to levels of DO
            \ddd Oxygen level changes result from streamflow, temperature, and run-off
            \ddd At sea level, \textbf{typical DO concentrations} in 100-percent saturated fresh water will range from $7.56 mg/L$ (or 7.56 parts oxygen in 1,000,000 parts water) at $30^\circ C$ to $14.62 mg/L$ at zero $0^\circ C$.
            \ddd Measured by the \vocab{Winkler Test}
            \\
    \mysection{Chemicals, Nutrients and Organisms}
        \mysubsection{E. Coli}
            \ddd These bacteria originate from the wastes of animals or humans. Thus, high numbers of E. coli in a pond could come from septic systems, runoff from barnyards, or from wildlife
            \ddd High levels of E. coli bacteria can be reduced by limiting animal access to the pond, maintaining nearby septic systems, and redirecting runoff
        \mysubsection{Nitrate-Nitrogen and Total Phosphorus}
            \ddd Nitrogen and phosphorus are nutrients that may cause increased growth of aquatic plants and algae.
            \ddd [N] > 3 mg/L and any detectable amounts of P may be indicative of fertilizers, manures, etc
        \mysubsection{Total Dissolved Solids (Sum of all chemical ions dissolved)}
            \ddd Controlled by natural source and nearby land activities (can be naturally high or low)
            \ddd Monitor relative changes
            \ddd Significant increases over time or >1000 mg/L indicates problem
        \mysubsection{pH}
            \ddd Pond or lake should fall btw 6 and 9; most fish do better around 7
            \ddd < 6 can stunt or eliminate fish pop.
            \ddd Inorganic buffers, such as sodium bicarbonate, are used to slow changes in pH → abundant in ocean and freshwater with limestone
            \ddd Low pH ponds are treated with limestone
        \mysubsection{Aluminum}
            \ddd Extremely toxic, [Al] > 0.1 mg/L toxic to most sensitive species like trout and minnows
            \ddd High levels from nearby coal mining or release from soils due to acid rain
            \ddd Causes fishes’ gills to become covered in thick mucus, so they can’t breathe
        \mysubsection{Sulfate}
            \ddd > 250 mg/L, from acid mine drainage or acid rain
            \ddd Usually combined with high levels of metals, high SO4 alone is only a problem when used for irrigation
        \mysubsection{Dissolved Oxygen}
            \ddd Must be measured quickly at pond (expensive) or less accurate (cheaper) kits
            \ddd Max DO controlled by temperature, pressure
            \ddd < 6 mg/L harmful to pond life
    \\
    \mysection{Analysis}
        properties include salinity, pH, alkalinity, phosphates, nitrates, turbidity, dissolved oxygen (DO), temperature, fecal coliform, total solids, and biological oxygen demand (BOD). 
        \mysubsection{Salinity}
            Water can be classified by its salinity as such: \textbf{fresh water has a ppt of $< 0.5$} which means that there are 0.5 molecules of dissolved salt for every 1000 molecules of solution, or 1 molecule of salt per every 2000 molecules of solution. Brackish water has a ppt between 0.5 and 30, saline water has a ppt between 30 to 50, and brine has a ppt of >50. The only water safe for human consumption is fresh water, and drinking water often achieves salinity levels as low as 0.1 ppt. In ocean water, total salt content makes up 3.5\% (35 ppt) of ocean water, the other 96.5\% being water. Of the dissolved salts in ocean water, 85\% of the salts is sodium chloride (3\% of all the water, 30 ppt). The other 15\% of the total salts are other salt ions such as Magnesium, Strontium etc. (.5 \% of total ocean water, 5 ppt). 
            \mysubsubsub{Consequences}
                Salinity in rivers and lakes of the US has recently been increasing due to road salt and other salt de-icers in runoff. \ddd It is very expensive to remove salt from water, and thus it is expensive to create drinking wate
        \mysubsection{pH}
            $pH = - \log [H^+]$
            \ddd  The normal pH of rivers in the United States is 6.5 to 8.5, and values between 6.0 and 9.0 can support life for fish and invertebrates. his makes acid rain an important factor in water quality, since it will make water more acidic
            \mysubsub{Influenced by} 
            \ddd \textbf{Human processes} like automobile/fossil fuel power plant emissions release nitrogen oxides and sulfur dioxide, which form acid once mixed with water
            \ddd  Coal mine drainage can lead to iron sulfide mixing with water and forming sulfuric acid.
            \ddd \textbf{Natural}. Limestone is a base when dissolved in water, so it can neutralize the effects of acids and increase the pH of the water. olcanoes, geysers, and hot springs will make water more acidic, as well as the presence of sulfur in nearby minerals.
        \mysubsection{Alkalinity}
            Alkalinity is the ability of a solution to neutralize an acid without changing the overall pH of the solution. Alkalinity arises from the presence of buffers in solutions, which have the properties to neutralize acids without changing the pH of a solution.  There are several ions that contribute to alkalinity, including bicarbonate, carbonate, hydroxide, and phosphate. Thus, limestone contributes to alkalinity, since its formula is calcium carbonate, and carbonate is one of the ions listed above. 
            \mysubsub{Effects}  Alkalinity is very important to water quality. Alkalinity in aquatic ecosystems needs to be within a certain range, depending upon the ecosystem. If alkalinity is too low, the ecosystem has low stability as it is susceptible to sudden pH changes from devices such as acid rain or other pollution, which can be harmful to the flora and fauna of the ecosystem. If alkalinity is too high, the buffer acids and bases in the buffer solution can render the ecosystem uninhabitable.
        \mysubsection{Phosphates}
            Phosphates can become a limiting nutrient in many systems, usually freshwater systems. Phosphates do not have too many ecological consequences. Generally, the only negative effect of an overabundance of phosphates is eutrophication. Phosphates can contribute to total dissolved solids. Extremely high levels of phosphates in \textbf{drinking water} can cause digestive issues. Phosphates are able to \textbf{enter waterways} in a variety of natural ways such as rocks or normal animal and plant waste in the water. \textbf{Human sources} such as fertilizers, pesticides, industrial and cleaning compounds, septic tanks and wastewater from sewage treatment can increase the amount of phosphates in the water. Phosphates are often released from mining and can enter waterways through mine runoffs. Erosion also significantly contributes to phosphate levels in bodies of water, as phosphates are abundantly found within soil and rock (see Phosphorous Cycle). The phosphates of a river or stream are usually measured in Parts Per Million (ppm).
        \mysubsection{Nitrates}
           Nitrates can also be a limiting nutrient, usually in marine systems rather than freshwater systems. However, high levels of nitrates in aquatic ecosystems can be detrimental to ecosystem health, inhibit the growth of some organisms, cause stress, and contribute to eutrophication. Nitrates do contribute to total dissolved solids and can be used as a water quality indicator. Nitrates are toxic to humans in even moderate concentrations, as they inhibit the oxygen flow through the body. \textbf{human sources} that can add to the total amount of nitrate in the water are fertilizers, poorly functioning septic tanks, inadequately treated wastewater from sewage treatment plants, manure from farm livestock, animal wastes including fish and birds, storm drains, runoff from crop fields, parks, lawns, feedlots and car exhausts. Most nitrates come from dead organisms and waste which releases ammonia, which is then oxidized to form nitrates. Nitrates are measured in Parts Per Million (ppm).
        \mysubsection{Turbidity}
            \ddd Nephelometric Turbidity Units (NTU) are also used. These are a measure of the tendency of particles to scatter a light beam that is focused on them. A nephelometer includes a light source shining at a column of water and a detector surrounding the column to the sides. The more light that reaches the detector, the more particles are in the water. To test the turbidity of the water, people also use a Secchi Disk. A marine Secchi disk is a plain white disk 30 cm in diameter and a freshwater Secchi disk is divided into fourths, two of which are white and two of which are black, and is 20 cm in diameter. The depth at which the disk is no longer visible is the measure of the turbidity of the water. Light can penetrate to a depth of about 2-3 times the Secchi Disk depth. 
            \ddd Seasonal variations can change the turbidity of a lake and lake turnover can also change it because of nutrients being released. Effects of turbidity include an increase in water temperature, decrease of photosynthetic rate, decreased growth, and more aesthetically displeasing water. Turbidity can also reduce the ability of fish gills to absorb dissolved oxygen. 
            \ddd The standard of turbidity for drinking water is 1 NTU at the plant outlet. All samples for turbidity must be less than or equal to 0.3 NTU for at least 95 percent. The World Health Organization establishes that the turbidity of drinking water should not be more than 5 NTU and should ideally be below 1 NTU. 
        \mysubsection{Dissolved Oxygen}
            \ddd DO is an important water quality indicator, as fish and other aerobic organisms require it for life. DO is usually measured in mg/L or ppm. These two quantities are equal. Percent dissolved oxygen is dependent upon many factors besides these, such as salinity and temperature.
            \ddd surface waters contain between 5 and 15 ppm of dissolved oxygen. If a stream or river has below 5 ppm of dissolved oxygen, then that can put aquatic life under stress, and below 1-2 ppm for a few hours can kill large fish living in the river.
            \ddd Supersaturation of oxygen (levels of DO over 100\%) can occur naturally through photosynthetically active species. Supersaturation can also occur through rapid changes in the environment that occur too quickly for the system to reach equilibrium, giving rise to DO levels over 100\% temporarily. 
        \mysubsection{Biochemical Oxygen Demand}
            \ddd (BOD) measures how fast organisms use up the oxygen in the water. Aerobic microbes use oxygen to oxidize the organic matter in the water, using the energy that is released in the process for growth and reproduction and creating a demand for DO.
            \ddd Most pristine rivers should have a 5-day BOD below 1 mg/L. Moderately polluted rivers may have a BOD between 2 and 8 mg/L. Municipal sewage that is treated with a three-stage process would have a BOD of about 20 mg/L. Untreated sewage has varying BOD but averages about 200 mg/L    
        \mysubsection{Temperature}
            \ddd water temperature in aquatic ecosystems is a very important quality indicator, as temperature affects other factors such as the dissolved oxygen level in the water, as well as photosynthesis of aquatic plants, metabolic rates of aquatic organisms, and more. Increases in the temperature of the water is called thermal pollution. Thermal pollution increases the sensitivity of organisms to disease, parasites, and pollution.
            \ddd removal of trees also means more sunlight hits streams. 
            \ddd  High temperature also decreases the ability of the water to hold DO, which has even more of an effect because temperature also increases the metabolic rates of aquatic organisms and their biochemical oxygen demand. 
            \ddd n very limited cases, an thermal pollution will have no effect or may even increase the ecosystem health. These cases are called thermal enrichment. An example of this is the manatee, which often uses power plant discharge sites during the winter.
        \mysubsection{Total Solids} of the water measures the suspended and dissolved solids in the body of water and is subdivided into those two categories: total suspended solids (TSS) and total dissolved solids (TDS). Sources of elevated levels of total solids may result as a result of runoff from agricultural activities, dredging, mining, salt from streets in winter, fertilizers from lawns, water treatment plants, plant materials, soil particles and soil erosion, and decaying organic matter.
        
        \mysection{EXTRA FROM BOTTOM OF BACK PAGW}
        water and running water. Females normally lay eggs on sides of containers filled with water (i.e. tires, flowerpots, natural holes) and the eggs require rainfall to rise the water level and trigger the larvae to hatch. Larvae are called wigglers, actively feeding in the water, siphoning organic matter floating around. Larval stage lasts 5-10 days, and pupal stage lasts 2 days. Life Cycle: Females require blood for egg development, and obtain it by sucking it out of its host(s) with an elongated proboscis. The bite is quick, so the reaction of a human attempting to swat it isn't fast enough to catch the fleeing mosquito. Males feed on nectar, sweet plant juices

\end{multicols*}
\end{document}